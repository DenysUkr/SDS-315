% Options for packages loaded elsewhere
\PassOptionsToPackage{unicode}{hyperref}
\PassOptionsToPackage{hyphens}{url}
%
\documentclass[
]{article}
\usepackage{amsmath,amssymb}
\usepackage{iftex}
\ifPDFTeX
  \usepackage[T1]{fontenc}
  \usepackage[utf8]{inputenc}
  \usepackage{textcomp} % provide euro and other symbols
\else % if luatex or xetex
  \usepackage{unicode-math} % this also loads fontspec
  \defaultfontfeatures{Scale=MatchLowercase}
  \defaultfontfeatures[\rmfamily]{Ligatures=TeX,Scale=1}
\fi
\usepackage{lmodern}
\ifPDFTeX\else
  % xetex/luatex font selection
\fi
% Use upquote if available, for straight quotes in verbatim environments
\IfFileExists{upquote.sty}{\usepackage{upquote}}{}
\IfFileExists{microtype.sty}{% use microtype if available
  \usepackage[]{microtype}
  \UseMicrotypeSet[protrusion]{basicmath} % disable protrusion for tt fonts
}{}
\makeatletter
\@ifundefined{KOMAClassName}{% if non-KOMA class
  \IfFileExists{parskip.sty}{%
    \usepackage{parskip}
  }{% else
    \setlength{\parindent}{0pt}
    \setlength{\parskip}{6pt plus 2pt minus 1pt}}
}{% if KOMA class
  \KOMAoptions{parskip=half}}
\makeatother
\usepackage{xcolor}
\usepackage[margin=1in]{geometry}
\usepackage{graphicx}
\makeatletter
\def\maxwidth{\ifdim\Gin@nat@width>\linewidth\linewidth\else\Gin@nat@width\fi}
\def\maxheight{\ifdim\Gin@nat@height>\textheight\textheight\else\Gin@nat@height\fi}
\makeatother
% Scale images if necessary, so that they will not overflow the page
% margins by default, and it is still possible to overwrite the defaults
% using explicit options in \includegraphics[width, height, ...]{}
\setkeys{Gin}{width=\maxwidth,height=\maxheight,keepaspectratio}
% Set default figure placement to htbp
\makeatletter
\def\fps@figure{htbp}
\makeatother
\setlength{\emergencystretch}{3em} % prevent overfull lines
\providecommand{\tightlist}{%
  \setlength{\itemsep}{0pt}\setlength{\parskip}{0pt}}
\setcounter{secnumdepth}{-\maxdimen} % remove section numbering
\usepackage{booktabs}
\usepackage{longtable}
\usepackage{array}
\usepackage{multirow}
\usepackage{wrapfig}
\usepackage{float}
\usepackage{colortbl}
\usepackage{pdflscape}
\usepackage{tabu}
\usepackage{threeparttable}
\usepackage{threeparttablex}
\usepackage[normalem]{ulem}
\usepackage{makecell}
\usepackage{xcolor}
\ifLuaTeX
  \usepackage{selnolig}  % disable illegal ligatures
\fi
\IfFileExists{bookmark.sty}{\usepackage{bookmark}}{\usepackage{hyperref}}
\IfFileExists{xurl.sty}{\usepackage{xurl}}{} % add URL line breaks if available
\urlstyle{same}
\hypersetup{
  pdftitle={HW 9},
  pdfauthor={Denys Osmak},
  hidelinks,
  pdfcreator={LaTeX via pandoc}}

\title{HW 9}
\author{Denys Osmak}
\date{2024-04-16}

\begin{document}
\maketitle

\hypertarget{get-out-the-vote}{%
\section{Get out the vote}\label{get-out-the-vote}}

\hypertarget{problem-1}{%
\section{Problem 1}\label{problem-1}}

\hypertarget{pt-a}{%
\subsection{Pt A}\label{pt-a}}

65\% of the people in the data did not vote in the 1998 election, while
45\% voted.

44\% of the people that did not receive the GOTV call, still voted in
1998 election

We can say with 95\% confidence that getting a GOTV phone call increased
the chances of voting 14-27\%

\hypertarget{pt-b}{%
\subsection{Pt B}\label{pt-b}}

The distribution of age is overall much higher for the people that voted
in 1998 election vs who didn't according to the box plot and the five
number summery. For example, the average age for someone who voted was
14 years older vs who didn't vote.

Next, 80\% of the people that are registered with a major party have
voted in the 1998 election vs 70\% of the people that are registered
with a major party and didn't voted. Additionally, 80\% of the people
that were registered with a party got the GOTV call, while 74\% of the
people registered with a party didn't get the GOTV call

Lastly, 76\% of the people that voted in the 1996 election also voted in
1998 election, while only 34\% of the people voted in the 1996 election
didn't vote in 1998 election. Next, 71\% of the people that voted in the
1996 election got GOTV call, while 53\% of the people voted in the 1996
election got GOVT call

\hypertarget{pt-c}{%
\subsection{Pt C}\label{pt-c}}

After matching the data, all of the people came from the same range of
years (19-96). Next the proportion of people that were registered with
majority party and voted and who didn't get the call matched to 80\% of
both subsets. Similarly, the proportion of people who voted in 1996
election \& voted and who didn't get the call matched to 71\% of both
subsets.

\hypertarget{what-do-you-conclude-about-the-overall-effect-of-the-gotv-call-on-the-likelihood-of-voting-in-the-1998-election}{%
\subsubsection{What do you conclude about the overall effect of the GOTV
call on the likelihood of voting in the 1998
election?}\label{what-do-you-conclude-about-the-overall-effect-of-the-gotv-call-on-the-likelihood-of-voting-in-the-1998-election}}

After matching the data to GOTV calls, confirming matching of the data,
and bootstrapping I can say with 95\% confidence (p value \textless{}
0.05) that getting a GOTV call increase the chance of someone voting by
0.5\%-14.1\%.

\hypertarget{problem-2}{%
\section{Problem 2}\label{problem-2}}

\hypertarget{pt-a-1}{%
\subsection{Pt A}\label{pt-a-1}}

\hypertarget{pt-b-1}{%
\subsection{Pt B}\label{pt-b-1}}

\hypertarget{pt-c-1}{%
\subsection{Pt C}\label{pt-c-1}}

\begin{itemize}
\item
  The baseline is having a Large Opening and a Thick alloy, represented
  by the intercept of having 0.393 skips
\item
  The main effect of having a Medium Opening is an increase of skips by
  2.4 compared to Having a Large Opening
\item
  The main effect of having a Small Opening is an increase of skips by
  5.1 compared to Having a Large Opening
\item
  The main effect of having a Thin Alloy is an increase of skips by 2.3
  compared to Having a Thick Alloy
\item
  The main effect of having a Medium Opening and a Thin Alloy is an
  decrease of skips by 0.7 compared to Large Opening + Thick Alloy
\item
  The main effect of having a Small Opening and a Thin Alloy is an
  increase of skips by 9.6 compared to Large Opening + Thick Alloy
\end{itemize}

\hypertarget{pt-d}{%
\subsection{Pt D}\label{pt-d}}

Based on the confidence intervals and the estimated weights of each
coefficient I would recommend sticking with Large Opening with Thick
alloy, however I would recommend further testing with Medium Opening and
Thin Alloy because according to the confidence interval it has potential
to decrease the amount of skips (but also to increase it)

\end{document}
