% Options for packages loaded elsewhere
\PassOptionsToPackage{unicode}{hyperref}
\PassOptionsToPackage{hyphens}{url}
%
\documentclass[
]{article}
\usepackage{amsmath,amssymb}
\usepackage{iftex}
\ifPDFTeX
  \usepackage[T1]{fontenc}
  \usepackage[utf8]{inputenc}
  \usepackage{textcomp} % provide euro and other symbols
\else % if luatex or xetex
  \usepackage{unicode-math} % this also loads fontspec
  \defaultfontfeatures{Scale=MatchLowercase}
  \defaultfontfeatures[\rmfamily]{Ligatures=TeX,Scale=1}
\fi
\usepackage{lmodern}
\ifPDFTeX\else
  % xetex/luatex font selection
\fi
% Use upquote if available, for straight quotes in verbatim environments
\IfFileExists{upquote.sty}{\usepackage{upquote}}{}
\IfFileExists{microtype.sty}{% use microtype if available
  \usepackage[]{microtype}
  \UseMicrotypeSet[protrusion]{basicmath} % disable protrusion for tt fonts
}{}
\makeatletter
\@ifundefined{KOMAClassName}{% if non-KOMA class
  \IfFileExists{parskip.sty}{%
    \usepackage{parskip}
  }{% else
    \setlength{\parindent}{0pt}
    \setlength{\parskip}{6pt plus 2pt minus 1pt}}
}{% if KOMA class
  \KOMAoptions{parskip=half}}
\makeatother
\usepackage{xcolor}
\usepackage[margin=1in]{geometry}
\usepackage{color}
\usepackage{fancyvrb}
\newcommand{\VerbBar}{|}
\newcommand{\VERB}{\Verb[commandchars=\\\{\}]}
\DefineVerbatimEnvironment{Highlighting}{Verbatim}{commandchars=\\\{\}}
% Add ',fontsize=\small' for more characters per line
\usepackage{framed}
\definecolor{shadecolor}{RGB}{248,248,248}
\newenvironment{Shaded}{\begin{snugshade}}{\end{snugshade}}
\newcommand{\AlertTok}[1]{\textcolor[rgb]{0.94,0.16,0.16}{#1}}
\newcommand{\AnnotationTok}[1]{\textcolor[rgb]{0.56,0.35,0.01}{\textbf{\textit{#1}}}}
\newcommand{\AttributeTok}[1]{\textcolor[rgb]{0.13,0.29,0.53}{#1}}
\newcommand{\BaseNTok}[1]{\textcolor[rgb]{0.00,0.00,0.81}{#1}}
\newcommand{\BuiltInTok}[1]{#1}
\newcommand{\CharTok}[1]{\textcolor[rgb]{0.31,0.60,0.02}{#1}}
\newcommand{\CommentTok}[1]{\textcolor[rgb]{0.56,0.35,0.01}{\textit{#1}}}
\newcommand{\CommentVarTok}[1]{\textcolor[rgb]{0.56,0.35,0.01}{\textbf{\textit{#1}}}}
\newcommand{\ConstantTok}[1]{\textcolor[rgb]{0.56,0.35,0.01}{#1}}
\newcommand{\ControlFlowTok}[1]{\textcolor[rgb]{0.13,0.29,0.53}{\textbf{#1}}}
\newcommand{\DataTypeTok}[1]{\textcolor[rgb]{0.13,0.29,0.53}{#1}}
\newcommand{\DecValTok}[1]{\textcolor[rgb]{0.00,0.00,0.81}{#1}}
\newcommand{\DocumentationTok}[1]{\textcolor[rgb]{0.56,0.35,0.01}{\textbf{\textit{#1}}}}
\newcommand{\ErrorTok}[1]{\textcolor[rgb]{0.64,0.00,0.00}{\textbf{#1}}}
\newcommand{\ExtensionTok}[1]{#1}
\newcommand{\FloatTok}[1]{\textcolor[rgb]{0.00,0.00,0.81}{#1}}
\newcommand{\FunctionTok}[1]{\textcolor[rgb]{0.13,0.29,0.53}{\textbf{#1}}}
\newcommand{\ImportTok}[1]{#1}
\newcommand{\InformationTok}[1]{\textcolor[rgb]{0.56,0.35,0.01}{\textbf{\textit{#1}}}}
\newcommand{\KeywordTok}[1]{\textcolor[rgb]{0.13,0.29,0.53}{\textbf{#1}}}
\newcommand{\NormalTok}[1]{#1}
\newcommand{\OperatorTok}[1]{\textcolor[rgb]{0.81,0.36,0.00}{\textbf{#1}}}
\newcommand{\OtherTok}[1]{\textcolor[rgb]{0.56,0.35,0.01}{#1}}
\newcommand{\PreprocessorTok}[1]{\textcolor[rgb]{0.56,0.35,0.01}{\textit{#1}}}
\newcommand{\RegionMarkerTok}[1]{#1}
\newcommand{\SpecialCharTok}[1]{\textcolor[rgb]{0.81,0.36,0.00}{\textbf{#1}}}
\newcommand{\SpecialStringTok}[1]{\textcolor[rgb]{0.31,0.60,0.02}{#1}}
\newcommand{\StringTok}[1]{\textcolor[rgb]{0.31,0.60,0.02}{#1}}
\newcommand{\VariableTok}[1]{\textcolor[rgb]{0.00,0.00,0.00}{#1}}
\newcommand{\VerbatimStringTok}[1]{\textcolor[rgb]{0.31,0.60,0.02}{#1}}
\newcommand{\WarningTok}[1]{\textcolor[rgb]{0.56,0.35,0.01}{\textbf{\textit{#1}}}}
\usepackage{graphicx}
\makeatletter
\def\maxwidth{\ifdim\Gin@nat@width>\linewidth\linewidth\else\Gin@nat@width\fi}
\def\maxheight{\ifdim\Gin@nat@height>\textheight\textheight\else\Gin@nat@height\fi}
\makeatother
% Scale images if necessary, so that they will not overflow the page
% margins by default, and it is still possible to overwrite the defaults
% using explicit options in \includegraphics[width, height, ...]{}
\setkeys{Gin}{width=\maxwidth,height=\maxheight,keepaspectratio}
% Set default figure placement to htbp
\makeatletter
\def\fps@figure{htbp}
\makeatother
\setlength{\emergencystretch}{3em} % prevent overfull lines
\providecommand{\tightlist}{%
  \setlength{\itemsep}{0pt}\setlength{\parskip}{0pt}}
\setcounter{secnumdepth}{-\maxdimen} % remove section numbering
\ifLuaTeX
  \usepackage{selnolig}  % disable illegal ligatures
\fi
\IfFileExists{bookmark.sty}{\usepackage{bookmark}}{\usepackage{hyperref}}
\IfFileExists{xurl.sty}{\usepackage{xurl}}{} % add URL line breaks if available
\urlstyle{same}
\hypersetup{
  pdftitle={HW 2},
  pdfauthor={Denys Osmak},
  hidelinks,
  pdfcreator={LaTeX via pandoc}}

\title{HW 2}
\author{Denys Osmak}
\date{2024-01-25}

\begin{document}
\maketitle

\hypertarget{problem-1}{%
\section{Problem 1}\label{problem-1}}

\hypertarget{part-a-create-a-histogram-to-display-the-overall-data-distribution-of-course-evaluation-scores.}{%
\subsection{Part A Create a histogram to display the overall data
distribution of course evaluation
scores.}\label{part-a-create-a-histogram-to-display-the-overall-data-distribution-of-course-evaluation-scores.}}

\begin{Shaded}
\begin{Highlighting}[]
\CommentTok{\#plot2()}
\end{Highlighting}
\end{Shaded}

\hypertarget{part-b-use-side-by-side-boxplots-to-show-the-distribution-of-course-evaluation-scores-by-whether-or-not-the-professor-is-a-native-english-speaker}{%
\subsection{Part B Use side-by-side boxplots to show the distribution of
course evaluation scores by whether or not the professor is a native
English
speaker}\label{part-b-use-side-by-side-boxplots-to-show-the-distribution-of-course-evaluation-scores-by-whether-or-not-the-professor-is-a-native-english-speaker}}

\hypertarget{part-c-use-a-faceted-histogram-with-two-rows-to-compare-the-distribution-of-course-evaluation-scores-for-male-and-female-instructors.}{%
\subsection{Part C Use a faceted histogram with two rows to compare the
distribution of course evaluation scores for male and female
instructors.}\label{part-c-use-a-faceted-histogram-with-two-rows-to-compare-the-distribution-of-course-evaluation-scores-for-male-and-female-instructors.}}

\hypertarget{part-d-create-a-scatterplot-to-visualize-the-extent-to-which-there-may-be-an-association-between-the-professors-physical-attractiveness-x-and-their-course-evaluations-y.}{%
\subsection{Part D Create a scatterplot to visualize the extent to which
there may be an association between the professor's physical
attractiveness (x) and their course evaluations
(y).}\label{part-d-create-a-scatterplot-to-visualize-the-extent-to-which-there-may-be-an-association-between-the-professors-physical-attractiveness-x-and-their-course-evaluations-y.}}

\hypertarget{problem-2}{%
\section{Problem 2}\label{problem-2}}

\hypertarget{part-a-a-line-graph-showing-average-hourly-bike-rentals-total-across-all-hours-of-the-day-hr.}{%
\subsection{Part A a line graph showing average hourly bike rentals
(total) across all hours of the day
(hr).}\label{part-a-a-line-graph-showing-average-hourly-bike-rentals-total-across-all-hours-of-the-day-hr.}}

\hypertarget{part-b-a-faceted-line-graph-showing-average-bike-rentals-by-hour-of-the-day-faceted-according-to-whether-it-is-a-working-day-workingday.}{%
\subsection{Part B a faceted line graph showing average bike rentals by
hour of the day, faceted according to whether it is a working day
(workingday).}\label{part-b-a-faceted-line-graph-showing-average-bike-rentals-by-hour-of-the-day-faceted-according-to-whether-it-is-a-working-day-workingday.}}

\hypertarget{part-c-a-faceted-bar-plot-showing-average-ridership-y-during-the-9-am-hour-by-weather-situation-code-weathersit-x-faceted-according-to-whether-it-is-a-working-day-or-not.-remember-that-you-can-focus-on-a-specific-subset-of-rows-of-a-data-set-using-filter.}{%
\subsection{Part C a faceted bar plot showing average ridership (y)
during the 9 AM hour by weather situation code (weathersit, x), faceted
according to whether it is a working day or not. (Remember that you can
focus on a specific subset of rows of a data set using
filter.)}\label{part-c-a-faceted-bar-plot-showing-average-ridership-y-during-the-9-am-hour-by-weather-situation-code-weathersit-x-faceted-according-to-whether-it-is-a-working-day-or-not.-remember-that-you-can-focus-on-a-specific-subset-of-rows-of-a-data-set-using-filter.}}

\hypertarget{problem-3}{%
\section{Problem 3}\label{problem-3}}

\hypertarget{part-a}{%
\subsection{Part A}\label{part-a}}

One faceted line graph that plots average boardings by hour of the day,
day of week, and month. You should facet by day of week. Each facet
should include three lines of average boardings (y) by hour of theday
(x), one line for each month and distinguished by color. Give the figure
an informative captionin which you explain what is shown in the figure
and also address the following questions, citing evidencefrom the
figure. Does the hour of peak boardings change from day to day, or is it
broadly similar acrossdays? Why do you think average boardings on
Mondays in September look lower, compared to otherdays and months?
Similarly, why do you think average boardings on Weds/Thurs/Fri in
Novemberlook lower? (Hint: wrangle first, then plot.)

\hypertarget{part-b}{%
\subsection{Part B}\label{part-b}}

One faceted scatter plot showing boardings (y) vs.~temperature (x),
faceted by hour of the day, and with points colored in according to
whether it is a weekday or weekend. Give the figure an informative
caption in which you explain what is shown in the figure and also answer
the following question, citing evidence from the figure. When we hold
hour of day and weekend status constant, does temperature seem to have a
noticeable effect on the number of UT students riding the bus?

\hypertarget{problem-4}{%
\section{Problem 4}\label{problem-4}}

\hypertarget{part-a-1}{%
\subsection{Part A}\label{part-a-1}}

Make a table of the top 10 most popular songs since 1958, as measured by
the total number of weeks that a song spent on the Billboard Top 100.
Note that these data end in week 22 of 2021, so the most popular songs
of 2021 onwards will not have up-to-the-minute data; please send our
apologies to The Weeknd.

Your table should have 10 rows and 3 columns: performer, song, and
count, where count represents the number of weeks that song appeared in
the Billboard Top 100. Make sure the entries are sorted in descending
order of the count variable, so that the more popular songs appear at
the top of the table. Give your table a short caption describing what is
shown in the table.

(Note: you'll want to use both performer and song in any group\_by
operations, to account for the fact that multiple unique songs can share
the same title.)

\hypertarget{part-b-1}{%
\subsection{Part B}\label{part-b-1}}

Is the ``musical diversity'' of the Billboard Top 100 changing over
time? Let's find out. We'll measure the musical diversity of given year
as the number of unique songs that appeared in the Billboard Top 100
that year. Make a line graph that plots this measure of musical
diversity over the years. The x axis should show the year, while the y
axis should show the number of unique songs appearing at any position on
the Billboard Top 100 chart in any week that year. For this part, please
filter the data set so that it excludes the years 1958 and 2021, since
we do not have complete data on either of those years. Give the figure
an informative caption in which you explain what is shown in the figure
and comment on any interesting trends you see.

There are number of ways to accomplish the data wrangling here. We offer
you two hints on two possibilities:

\begin{enumerate}
\def\labelenumi{\arabic{enumi})}
\item
  You could use two distinct sets of data-wrangling steps. The first set
  of steps would get you a table that counts the number of times that a
  given song appears on the Top 100 in a given year. The second set of
  steps operate on the result of the first set of steps; it would count
  the number of unique songs that appeared on the Top 100 in each year,
  irrespective of how many times it had appeared.
\item
  You could use a single set of data-wrangling steps that combines the
  length and unique commands.
\end{enumerate}

\hypertarget{part-c}{%
\subsection{Part C}\label{part-c}}

Let's define a ``ten-week hit'' as a single song that appeared on the
Billboard Top 100 for at least ten weeks. There are 19 artists in U.S.
musical history since 1958 who have had at least 30 songs that were
``ten-week hits.'' Make a bar plot for these 19 artists, showing how
many ten-week hits each one had in their musical career. Give the plot
an informative caption in which you explain what is shown

\hypertarget{part-d}{%
\subsection{Part D}\label{part-d}}

\end{document}
